\chapter{Комбинация камеры и радара в автономных системах}

На сегодняшний день в системах отслеживания объектов использование камеры является 
одним из самых популярным инструментом, однако есть ряд проблем с которым сталкиваются
инженеры при использовании камеры в системах автономных автомобилей:

\begin{itemize}

	\item Освещение (ночь, тень, яркий свет);
	
	\item погодные условия (дождь, снег, туман);
	
	\item ограниченные возможности определения точных расстояний и скоростей.

\end{itemize}

Даже при всех плюсах камеры (высокая детализация изображения, возможность распознавания
и классификации объектов, восприятие цветовой информации), минусы представляют из себя
довольно серьезные и интересные задачи.

Для нивелирования недостатков камеры, часто используют радары. Они измеряют расстояние
до объектов и их скорость с высокой точностью, особенно в условиях, когда другие
сенсоры могут дать неточные данные. Радары работают на радиоволнах, что делает их
нечувствительными к условиям освещения и погодным факторам (таким как дождь, туман,
темнота, яркий свет). К недостаткам радаров можно отнести недостаточность информации 
о форме объектов и их визуальных особенностях, а их низкая детализация.

Совмещение данных с камер и радаров позволяет создать более полную картину окружающей
среды, благодаря чему система отслеживания может более точно и надежно идентифицировать
и отслеживать объекты. При использовании Sensor Fusion система объединяет информацию от
камер и радаров для повышения точности и надежности восприятия \cite{Wang2020}. Камеры предоставляют
визуальные данные о форме и идентификации объектов, а радары дают точную информацию о расстоянии до объектов и их скорости. Таким образом компенсируются общие недостатки: 
камеры могут плохо работать в условиях низкой освещенности или при плохой погоде, тогда
как радары остаются эффективными. В таких условиях данные с радаров могут компенсировать
ограничения камер. С другой стороны, если объект сложно идентифицировать по данным 
радара, камеры могут предоставить недостающую информацию.

Также, в случае использования камер и радаров, можно использовать систему приоритетов:
в зависимости от условий окружающей среды, система автономного вождения динамически 
решает данные с какого источника являются более надежными и достоверными в в текущей 
ситуации и на основании этого корректируют своё поведение.

Таким образом, использование камер и радаров в комплексе позволяет существенно повысить безопасность и точность системы отслеживания объектов в автономных транспортных средствах. Эти сенсоры дополняют друг друга, компенсируя слабости одного за счет сильных сторон другого, обеспечивая надежное восприятие окружающей среды и позволяя системе принимать верные решения в сложных условиях.
