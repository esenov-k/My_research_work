\chapter{Применение фильтра Калмана}

Фильтр Калмана — один из ключевых методов для оценки положения и скорости движущихся 
объектов. Фильтр Калмана представляет из себя рекурсивный алгоритм, который используется 
для оценки состояния системы (в нашем случае положение объекта) на основе 
последовательности шумных измерений (с камер и радаров) \cite{KF}.

Основной проблемой всех датчиков и данных, которые с них снимаются, является шум измерений
который обусловлен многими факторами, на которые невозможно повлиять человеку. Это значит, 
что фильтр должен находить баланс между шумом измерения и ошибкой предсказания.

При этом, чем дальше прогнозируется будущее, тем меньше уверенности в прогнозе и тем больше
доверия к измерениям. Однако прогнозы на более короткие периоды обрабатываются с большей
уверенностью и могут быть использованы для оспаривания результатов измерений.
Говоря простым языком, мы имеем две неопределённости – прогноза и измерения, и нам надо
выбрать такие уровни доверия к ним, чтобы оценка положения находилась в оптимальном балансе. 

Например, если измерить положение объекта несколько раз при помощи шумящего датчика,
результаты будут распределены случайным образом. В идеальной ситуации, когда среднее 
значение шума равно нулю, измерения будут сгруппированы вокруг истинного состояния 
объекта. Однако в реальных условиях, например, в радарных системах, шум может увеличиваться
по мере роста расстояния, что требует более сложных моделей учета шума.

\section{Основные этапы работы фильтра Калмана}

Работу фильтра Калмана в целом можно разделить на 3 этапа:

\begin{enumerate}

    \item \textbf{Предсказание состояния}: на основе предыдущего состояния и динамической модели объекта предсказывается текущее положение.
	
    \item \textbf{Обновление измерений}: на основе текущих измерений с сенсоров корректируется предсказанное состояние.
	
    \item \textbf{Оценка точности}: на каждом шаге фильтр корректирует как предсказанное состояние, так и ковариацию ошибок.
	
\end{enumerate}

\section{Математические описание Калмана:}

\begin{itemize}
    \item Пусть \( \mathbf{x}_k \) — вектор состояния объекта в момент времени \( k \), включающий его положение и скорость, например:
    \[
    \mathbf{x}_k = \begin{bmatrix} x \\ y \\ v_x \\ v_y \end{bmatrix},
    \]
    где \( x, y \) — координаты объекта, \( v_x, v_y \) — скорости вдоль осей.

    \item \textbf{Предсказание нового состояния}:
    \[
    \mathbf{x}_k = \mathbf{A} \mathbf{x}_{k-1} + \mathbf{B} \mathbf{u}_{k-1} + 
	\mathbf{w}_{k-1},
    \]
    где \( \mathbf{A} \) — матрица перехода состояния, \( \mathbf{B} \) — матрица 
	управления, \( \mathbf{u}_{k-1} \) — вектор управляющих воздействий, 
	\( \mathbf{w}_{k-1} \) — шум модели.
    
    \item \textbf{Предсказание ковариации ошибок}:
    \[
    \mathbf{P}_k = \mathbf{A} \mathbf{P}_{k-1} \mathbf{A}^T + \mathbf{Q},
    \]
    где \( \mathbf{P}_k \) — ковариационная матрица ошибок, \( \mathbf{Q} \) — шум процесса.

    
    \item \textbf{Обновление с учётом измерений}:
    \[
    \mathbf{K}_k = \mathbf{P}_k \mathbf{H}^T (\mathbf{H} \mathbf{P}_k \mathbf{H}^T + \mathbf{R})^{-1},
    \]
    где \( \mathbf{K}_k \) — матрица Калмана, \( \mathbf{H} \) — матрица измерений, 
	\( \mathbf{R} \) — ковариация ошибок измерений.
    
    \item \textbf{Коррекция состояния}:
    \[
    \mathbf{x}_k = \mathbf{x}_k + \mathbf{K}_k (\mathbf{z}_k - \mathbf{H} \mathbf{x}_k),
    \]
    где \( \mathbf{z}_k \) — вектор измерений.
    
    \item \textbf{Обновление ковариации ошибок}:
    \[
    \mathbf{P}_k = (\mathbf{I} - \mathbf{K}_k \mathbf{H}) \mathbf{P}_k.
    \]
\end{itemize}

\section{Сложности использования фильтра Калмана}

Одна из сложностей заключается в том, что стандартные фильтры Калмана могут точно 
моделировать только простые движения, такие как движение с постоянной скоростью, 
ускорением или поворотом. Более сложные траектории часто требуют комбинации нескольких 
моделей движения. Например, поворот с большим радиусом может быть описан как сочетание 
движения по прямой и поворота с меньшим радиусом. Чтобы учесть такие сложности, 
применяются методы одновременного использования нескольких моделей прогнозирования.

В этом случае можно использовать:

\begin{itemize}

    \item расширенный фильтр Калмана;
	
	\item сигма-точечный фильтр Калмана;
	
	\item метод взаимодействующих множественных моделей (IMM).
	
\end{itemize}

В первых двух случаях фильтр Калмана будет учитывать нелинейности модели, что улучшит точность
предсказаний. Однако такие модели довольно ресурсоёмкие. 

Метод взаимодействующих множественных моделей представляет из себя несколько фильтров Калмана
которые  работают параллельно, при каждая модель предсказывает состояние объекта на основе 
разных сценариев движения \cite{Uss}. После поступления новых данных от датчика алгоритм сравнивает 
предсказания всех моделей и выбирает ту, которая наиболее точно описывает реальное движение. 
Веса моделей пересчитываются по мере поступления новых данных, что позволяет корректировать 
не только основную модель, но и другие, снижая их ошибки. Но использование такого метода 
также является ресурсоёмким, поэтому в реальных системах приходится выбирать ограниченное
количество моделей для оптимальной работы.

Кроме того, когда в зоне наблюдения несколько объектов, возникает проблема привязки данных.
Измерения могут быть неправильно привязаны к объектам, особенно если объекты находятся близко 
друг к другу. Это может привести к тому, что измерения одного объекта будут использоваться для 
корректировки состояния другого, что неприемлемо для точного слежения.

Таким образом, применение фильтра Калмана и его вариантов для отслеживания объектов в 
автономных системах управления транспортом является эффективным способом решения задачи 
слежения, однако необходимо учитывать возможные ограничения линейной модели и шума измерений. 
Комбинирование с многосенсорными данными и адаптивными методами позволяет улучшить 
производительность таких систем.
