\chapter*{ЗАКЛЮЧЕНИЕ}
\addcontentsline{toc}{chapter}{ЗАКЛЮЧЕНИЕ}

В ходе практики было выявлено, что дополнительное использование радаров в системах
отслеживания объектов поможет избавиться от недостатков камер и в целом улучшить 
точность системы. Прогнозирование траектории, которое производится при помощи 
фильтра Калмана является эффективным способом решения задачи слежения, однако 
необходимо учитывать возможные ограничения линейной модели и шума измерений.
Комбинирование с многосенсорными данными и адаптивными методами позволяет
улучшить производительность таких систем. Реализация фильтра Калмана на языке C++
позволит интегрировать его в системы отслеживания объектов. Однако в дальнейшем
программная реализация будет дорабатываться под конкретные задачи которые будут
стоять в рамках ВКРМ.