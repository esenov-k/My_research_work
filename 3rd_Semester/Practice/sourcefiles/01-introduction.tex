\chapter*{ВВЕДЕНИЕ}
\addcontentsline{toc}{chapter}{ВВЕДЕНИЕ}

Применение технологий слежения за объектами используется во многих высокотехнологичных 
системах, особенно в системах целеуказания и автономных системах. Такие системы должны
отслеживать движения объектов в требуемой зоне без участия оператора.

Данная работа реализована в рамках производственной практики кафедры СМ-7 
"Робототехнические системы и мехатроника" МГТУ им. Н.Э. Баумана.

Главной целью летней практики является анализ и совершенствование технологии отслеживания
объектов, а также предсказания их дальнейшей траектории с использованием фильтра Калмана.

Задачи, решаемые в рамках производственной практики следующие:

\begin{enumerate}

	\item \textbf{Анализ необходимости использования совместной работы камеры и радара.}
	Исследование того, как камеры и радары дополняют друг друга, какие преимущества
	их совместной работы важны для систем автономного вождения. 
	\item \textbf{Применение и настройка фильтра Калмана.}Реализация и адаптация
	фильтра Калмана для системы отслеживания объектов, с учётом особенностей данных
	от камер и радаров.
	\item \textbf{Программная реализация на языке C++}

\end{enumerate}