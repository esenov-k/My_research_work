\chapter{Адаптивный круиз-контроль}
Адаптивный круиз-контроль --- это продвинутая версия традиционного круиз-контроля, который поддерживает не только заданную скорость, но и автоматически регулирует скорость автомобиля для поддержания безопасного расстояния до впереди идущего автомобиля.

В основе работы простого АКК лежат следующие основные части: датчики, которые отвечают за определение расстояние впереди движущегося автомобиля, блок управления и исполнительные устройства (двигатель, КПП, тормозная система). В отличие от обычного круиз-контроля, адаптивный способен подстраиваться под окружающий трафик, ускоряя машину до заданной скорости, либо замедляя ее до скорости впереди идущего автомобиля, вплоть до полной остановки.

При введении МОТ в систему АКК, решается ряд следующих задач:

\begin{itemize}

	\item Обнаружение и Отслеживание --- MOT используется для обнаружения и отслеживания положения и скорости транспортных средств впереди. Это достигается с помощью радаров, лидаров или камер, которые непрерывно анализируют дорожную обстановку.

	\item Принятие Решений --- Система АКК получает данные от MOT и принимает решение о необходимости ускорения или торможения. Если расстояние до впереди идущего транспортного средства уменьшается, система автоматически снизит скорость, чтобы поддерживать безопасный интервал.

	\item Прогнозирование Траекторий --- MOT также может помогать в прогнозировании траекторий движения окружающих транспортных средств, что позволяет системе АКК предвидеть возможные изменения в движении и заранее адаптироваться к ним.

\end{itemize}

За счет того, что в МОТ используются как камеры, так и лидары/радары, можно говорить о более точной работе системы АКК. Анализ окружающей среды при помощи МОТ и дальнейшей его обработка в блоке управления позволят улучшить параметры точности системы и сведут возможную ошибку к минимуму.  

\chapter{Мультиобъектный трекинг (МОТ)}

Мультиобъектный трекинг (MOT) — это процесс идентификации и отслеживания нескольких объектов в пространстве и времени, при помощи компьютерного зрения. MOT важен для понимания и анализа сложных сцен, где множество объектов взаимодействуют друг с другом, и он находит применение во многих областях, включая видеонаблюдение, робототехнику и, конечно же, автономные транспортные средства.

В соответствии с SAE International существует 6 уровней автоматизации автономных автомобилей [ссылка]:

\begin{itemize}

	\item 	0 уровень --- отсутствие автомотизации. Системы нулевого уровня осуществляют только функции предупреждения водителя об опасной ситуации, за весь процесс вождения отвечает водитель

	\item 	1 уровень --- помощь водителю. В системах первого уровня часть функций управления автомобиля осуществляет автоматика, при этом водитель постоянно находится в готовности взять полное управление автомобилем на себя;
	
	\item	2 уровень --- частичная автомотизация. В таких системах автоматика полностью управляет автомобилем в ряде ситуаций, при этом водитель находит ся в готовности взять управление на себя в случае, если система не справляется;
	
	\item	3 уровень --- условная автомотизация. При определенных обстоятельствах система может сама выполнять все функции управления, но водитель все ещё должен быть готов взять на себя управление, когда система не способна выполнять свои функции;
	
	\item 	4 уровень --- высокая автомотизация. ТС может двигаться самостоятельно почти во всех ситуациях, но нестандартные дорожные ситуации или другие внешние факторы могут потребовать вмешательства человека, при этом человек может сидеть в пассажирском кресле;
	
	\item 	5 уровень --- полная автомотизация. ТС способно выполнять все функции управления в любых обстоятельствах без человеческого вмешательства.

\end{itemize}

МОТ может быть применим ко всем уровням автономности в той или иной степени. Например, для уровней с нулевого по второй MOT может использоваться для функций, таких как адаптивный круиз-контроль и предотвращение столкновений, а уже для уровней с трейтьего и выше, MOT является критически важной составляющей, обеспечивающей осведомленность систем автомобиля о динамической дорожной обстановке.


