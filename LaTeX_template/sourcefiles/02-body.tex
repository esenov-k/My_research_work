\chapter{Адаптивный круиз-контроль}
Адаптивный круиз-контроль --- это продвинутая версия традиционного круиз-контроля, который поддерживает не только заданную скорость, но и автоматически регулирует скорость автомобиля для поддержания безопасного расстояния до впереди идущего автомобиля.

В основе работы простого АКК лежат следующие основные части: датчики, которые отвечают за определение расстояние впереди движущегося автомобиля, блок управления и исполнительные устройства (двигатель, КПП, тормозная система). В отличие от обычного круиз-контроля, адаптивный способен подстраиваться под окружающий трафик, ускоряя машину до заданной скорости, либо замедляя ее до скорости впереди идущего автомобиля, вплоть до полной остановки.

При введении МОТ в систему АКК, решается ряд следующих задач:

\begin{itemize}

	\item Обнаружение и Отслеживание --- MOT используется для обнаружения и отслеживания положения и скорости транспортных средств впереди. Это достигается с помощью радаров, лидаров или камер, которые непрерывно анализируют дорожную обстановку.

	\item Принятие Решений --- Система АКК получает данные от MOT и принимает решение о необходимости ускорения или торможения. Если расстояние до впереди идущего транспортного средства уменьшается, система автоматически снизит скорость, чтобы поддерживать безопасный интервал.

	\item Прогнозирование Траекторий --- MOT также может помогать в прогнозировании траекторий движения окружающих транспортных средств, что позволяет системе АКК предвидеть возможные изменения в движении и заранее адаптироваться к ним.

\end{itemize}

За счет того, что в МОТ используются как камеры, так и лидары/радары, можно говорить о более точной работе системы АКК. Анализ окружающей среды при помощи МОТ и дальнейшей его обработка в блоке управления позволят улучшить параметры точности системы и сведут возможную ошибку к минимуму.  