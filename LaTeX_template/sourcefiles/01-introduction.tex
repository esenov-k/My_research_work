\chapter*{ВВЕДЕНИЕ}
\addcontentsline{toc}{chapter}{ВВЕДЕНИЕ}

В последние годы значительно возрос интерес к развитию интеллектуальных транспортных систем, целью которых является повышение безопасности и эффективности дорожного движения. Важным аспектом в этом направлении является внедрение продвинутых систем помощи водителям, таких как адаптивный круиз-контроль (АКК) и автоматическое экстренное торможение (АЭТ). Эти системы способны не только существенно снизить количество дорожно-транспортных происшествий, но и сделать процесс вождения более комфортным и менее утомительным. Центральное место в разработке таких систем занимает мультиобъектный трекинг (MOT), который позволяет одновременно отслеживать множество объектов в поле зрения транспортного средства.

Системы адаптивного круиз-контроля используют MOT для обеспечения автоматического поддержания безопасного расстояния до впереди идущего автомобиля. С помощью датчиков и камер, MOT анализирует траектории движения окружающих автомобилей, позволяя AКК адаптироваться к изменяющимся дорожным условиям, таким как замедление или ускорение трафика. В свою очередь, системы автоматического торможения реагируют на непредвиденные препятствия, такие как пешеходы или внезапно останавливающиеся транспортные средства, предотвращая столкновения или снижая их тяжесть благодаря быстрому и точному отслеживанию положения и скорости потенциальных препятствий.

Точность MOT в таких системах имеет решающее значение, поскольку ошибки в отслеживании могут привести к неправильному распознаванию объектов или их траекторий, что в свою очередь может вызвать нежелательное активирование тормозов или неправильное регулирование скорости. Поэтому важным направлением исследований в области автомобилестроения является улучшение алгоритмов MOT, чтобы повысить их эффективность и надежность в разнообразных и динамически меняющихся дорожных сценариях.

Применение MOT не ограничивается только наземными транспортными средствами. Дроны, автономные летательные аппараты и робототехника также полагаются на эффективное отслеживание объектов для выполнения миссий поиска и спасания, наблюдения и многих других задач. С учетом этого, настоящее исследование направлено на анализ существующих методов MOT и разработку новых подходов, которые могли бы обеспечить более высокую точность и скорость отслеживания в системах AКК и AЭТ. Особое внимание будет уделено алгоритмам машинного обучения и искусственного интеллекта, способным адаптироваться к переменчивым условиям эксплуатации и обеспечить высокую степень автономности транспортных средств.

В ходе данной работы предполагается использовать комплексный подход, сочетающий теоретические исследования алгоритмов MOT, их программную реализацию и последующее тестирование в имитационных и реальных условиях. Целью является не только улучшение существующих функциональных возможностей AКК и AЭТ, но и создание основы для будущих инноваций в области автомобилестроения и {робототехники}.
