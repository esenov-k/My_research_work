\chapter*{ВВЕДЕНИЕ}
\addcontentsline{toc}{chapter}{ВВЕДЕНИЕ}

В современном мире серверные системы играют ключевую роль в обеспечении стабильной работы 
различных сервисов и приложений. Одним из критически важных проблем является эффективное 
охлаждение серверного оборудования. Высокая плотность размещения вычислительных модулей и 
постоянное увеличение их производительности приводят к значительному выделению тепла, что 
может негативно сказаться на надежности и сроке службы оборудования. Недостаточное охлаждение 
может вызвать перегрев, что в свою очередь приведет к сбоям в работе серверов, снижению их 
производительности и даже к потере данных.

Актуальность темы обусловлена ростом количества серверных ферм и центров обработки данных (ЦОД), а также увеличением требований к их надежности и энергоэффективности. Традиционные методы охлаждения серверов, такие как использование статических вентиляторов или систем жидкостного охлаждения, зачастую оказываются недостаточно гибкими и энергоэффективными. В этой связи актуальной задачей является разработка интеллектуальной системы управления кулерами, способной динамически регулировать скорость вращения вентиляторов в зависимости от текущих температурных условий внутри серверного шкафа.

Целью данного проекта является разработка платы управления кулерами, которая сможет автоматически регулировать скорость вращения вентиляторов в зависимости от заданной температуры, обеспечивая тем самым оптимальные условия работы серверного оборудования.