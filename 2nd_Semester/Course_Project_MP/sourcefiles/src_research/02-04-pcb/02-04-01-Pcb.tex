При проектировании печатной платы необходимо учитывать следующие параметры:

\begin{enumerate}

	\item Длина, ширина и толщина печатной платы;
	
	\item Тип платы (односторонняя, двусторонняя, многослойная);
	\item Минимальная ширина проводников силовой цепи;
	
	\item Минимальное расстояние между проводниками.
	
	\item Толщина проводников платы.

\end{enumerate}

Важно иметь ввиду, что необходимо рационально располагать все компоненты печатной с учетом их функционального назначения, а также с учетом особенностей подключения. 

Согласно ГОСТ P 53429-2009 \cite{doc::Gost53429} при длине платы менее 10 мм, размеры с каждой стороны должны быть кратными 2.5~мм.

Минимальная ширина проводников силовой цепи рассчитывается по формуле из ГОСТ 23751-86 \cite{doc::Gost23751}:

\[
t_{min}>\frac{I_{max}}{175\times h} = \frac{1}{175*35} = 0.16 \text{мм}
\]
Таким образом минимальная толщина дорожки должна быть 0.16~мм.

Трассировка проводилась с использованием комплексной системы автоматизированного проектирования Altium Designer 2021, которая обеспечивает полный цикл проектирования - от создания электрической схемы до выпуска конструкторской документации для разработанного устройства.