\section {Подключение питания}

В качестве источника напряжения присутствует блок питания на 12 В. Для работы датчика, МК, дисплея и всех остальных элементов, необходимо 3.3 В. Для вентиляторов необходимо 12В. Таким образом, необходимо использовать один преобразователь напряжения с 12 до 3.3 В. 

Для этой цели подходит преобразователь напряжения LM2675, характеристики которого представлены в таблице \ref{tab::DcDcConverter}

\begin{table}[h!]
	\centering
	\caption{Характеристики регулятора напряжения LM2675}
	\begin{tabular}{| m{6cm} | m{8cm} |}
		\hline
		\textbf{Характеристика} & \textbf{Значение} \\ \hline
		КПД & до 96\% \\ \hline
		Доступные корпуса & 8-pin SOIC и PDIP, 16-pin WSON \\ \hline
		Выходные напряжения & 3.3В, 5.0В, 12В \\ \hline
		Максимальная выходная нагрузка & 1A \\ \hline
		Диапазон входного напряжения & 8В до 40В \\ \hline
		Ток покоя & 2.5 мА \\ \hline
		Ток в режиме ожидания & 50 мкА \\ \hline
		Сопротивление переключателя & 0.25 Ом \\ \hline
		Защита от перегрева и ограничения тока & Включена \\ \hline
		Температурный диапазон & от -40°C до +125°C \\ \hline
	\end{tabular}
	\label{tab::DcDcConverter}
\end{table}

Схема подключения преобразователя напряжения представлена на рисунке \ref{fig::DcDcConnection}

\img[htb]{fig::DcDcConnection}{DcDcConv_Connection.png}{Схема подключения LM2675}{0.6}