\section {Подключение питания}

В качестве источника напряжения присутствует блок питания на 12 В. Для подключения питания к плате воспользуемся винтовым клеммником на два входа KLS2-125-3.81-02P-4S \cite{datasheet::DcDc}, с рабочим напряжением в 300~В и током в 10~А. Этих значений хватит с запасом для нашей платы. 

Для работы датчика, МК, дисплея и всех остальных элементов, необходимо 3.3 В. Для вентиляторов необходимо 12В. Таким образом, необходимо использовать один преобразователь напряжения с 12 до 3.3 В. 

Для этой цели подходит преобразователь напряжения N7803 C-типа, схема подключения которого представлена на рисунке \ref{fig::DcDcConnection}, а характеристики в таблице \ref{tab::DcDcConverter}.

\img[h!]{fig::DcDcConnection}{DcDcConv_Connection.png}{Схема подключения N7803}{0.6}

\begin{table}[H]
	\centering
	\caption{Основные характеристики регулятора напряжения серии N78}
	\begin{tabular}{| m{6cm} | m{8cm} |}
		\hline
		\textbf{Характеристика} & \textbf{Значение} \\ \hline
		Эффективность & До 96\% \\ \hline
		Диапазон входного напряжения & 6 - 36~В \\ \hline
		Номинальное выходное напряжение & 3.3~В \\ \hline
		Максимальный выходной ток & 1 А \\ \hline
		Диапазон рабочих температур & -40°C до +85°C \\ \hline
		Максимальное напряжение на входе & 40 В \\ \hline
		Частота переключения & 500 кГц \\ \hline
		Ток покоя & 6 - 24 мА \\ \hline
		Защита & Защита от короткого замыкания, перегрузки и перегрева \\ \hline
	\end{tabular}
	\label{tab::DcDcConverter}
\end{table}

Размеры преобразователя C-типа: 11.5 мм x 9.0 мм x 17.5 мм.

Ci и Co выберем исходя из спецификации: 10 мкФ и 22 мкФ, соответственно.

