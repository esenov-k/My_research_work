\section {Подключение микроконтроллера}

\subsection {Подключение питания}

Корректная работа МК во многом зависит от правильного подключения питания. На рисунке \ref{fig::ESP32}
изображено расположение контактов МК ESP32-D0WD-V3.

\img[htb]{fig::ESP32}{ESP32_Pins.png}{Рапиновка МК ESP32-D0WD-V3}{0.8}

Подключение напряжения питания будем осуществлять по паспорту МК \cite{datasheet::ESP32}. Для пинов питания необходимо подать напряжение в диапазоне 2.3-3.6~В, GND подключить к общему проводу и для фильтрации требуется следующее подключение: 

\begin{itemize}

	\item Для контакта питания VDDA (пин 1) необходим конденсатор емкостью 0.1~мкФ (GRM319R72A104KA01D);
	
	\item Для VDDA (пины 46 и 43) необходимы 2 конденсатора на 100~пФ (GRM1555C1H101GA01D) и 1~мкФ (GRM21BR71C105K);
	
	\item Для пинов питания VDD3P3 (пины 3 и 4) необходимо подключение согласно рисунку \ref{fig::VDD3P3}. Конденсаторы, которые обозначены NC (not connected) не требуются в нашем случае. Выберем следующие компоненты: конденсатор емкостью 10~мкФ --- GRM188R61C106KAALD; конденсатор емкостью 1~мкФ --- GRM21BR71C105K; конденсатор емкостью0.1~мкФ --- GRM319R72A104KA01D; катушка индуктивностью 2 нГн --- LQP15TN2N0C02D.

\end{itemize}

\img[htb]{fig::VDD3P3}{VDD3P3.png}{Подключение питания VDD3P3}{0.5}

\subsection{Подключение кнопки включения и сброса}

Для реализации кнопки включения и сброса системы необходимо использовать контакт CHIP\_PU. Чтобы обеспечить правильное время включения и сброса рекомендуется использовать RC цепочку со следующими характеристиками: делитель напряжения из резисторов номиналом 10 кОм (0402WGF1002TCE) --- для подтяжки питания и 470 Ом (CRCW0805470RFKEA) --- для защиты от дребезга; конденсатор для защиты от помех емкостью 1нФ (GRM21A7U2E102JW31D); кнопка IT-0350. Схема включения представлена на рисунке \ref{fig::ResetButtonConnection} 

\img[htb]{fig::ResetButtonConnection}{Button_Connection.png}{Подключение кнопки сброса}{0.5}

\subsection{Подключение кварцевого резонатора}

Для стабильной работы МК требуется подключение внешнего кварцевого резонатора с частотой 40~МГц и точностью $\pm$10~ppm. Для этой цели выберем кварцевый резонатор ECX-1637B2 \cite{datasheet::CrystalClock}, который удовлетворяет всем необходимым требованиям по частоте и по точности. Схема подключения указана на рисунке \ref{fig::CrystalClockConnection}.

\img[htb]{fig::CrystalClockConnection}{CrystalClockConnection.png}{Подключение кварцевого резонатора ECX-1637B2}{0.5}

Для расчета номинала конденсаторов воспользуемся следующей формулой:

\[
C_L = \frac{C1 \times C2}{C1 + C2} + C_{stray}
\]

где:
\begin{itemize}
	\item \( C_L \) — нагрузочная емкость, 8 пФ исходя из спецификации резонатора;
	\item \( C_{stray} \) — паразитная емкость платы, примем 5 пФ;
	\item \( C1 \) и \( C2 \) — значения конденсаторов, подключенных к выводам кварцевого резонатора.
\end{itemize}

Так как обычно выбирают одинаковые номиналы конденсаторов, то для \( C1 = C2 = C \):

\[
8 \text{ пФ} = \frac{C \times C}{C + C} + 5 \text{ пФ}
\]

\[
8 \text{ пФ} = \frac{C^2}{2C} + 5 \text{ пФ}
\]

\[
8 \text{ пФ} = \frac{C}{2} + 5 \text{ пФ}
\]

\[
3 \text{ пФ} = \frac{C}{2}
\]

\[
C = 6 \text{ пФ}
\]:

Таким образом, для подключения кварцевого резонатора необходимо использовать конденсаторы номиналом 6~пФ (06035A6R0CAT2A).

Также, рекомендуется соединить контакты CAP1 и СAP2 согласно паспорту МК \cite{datasheet::ESP32}. Это делается для стабилизации работы МК. Схема соединения представлена на рисунке \ref{fig::CAP1CAP2}

\img[htb]{fig::CAP1CAP2}{CAP1CAP2.png}{Подключение контактов CAP1 и CAP2}{0.4}