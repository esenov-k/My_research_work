\section {Подключение микроконтроллера}

Корректная работа МК во многом зависит от правильного подключения питания. На рисунке \ref{fig::ESP23}
изображено расположение контактов МК ESP32-D0WD-V3.

\img[htb]{fig::ESP32}{ESP32_Pins.png}{Рапиновка МК ESP32-D0WD-V3}{0.8}

Подключение напряжения питания будем осуществлять по паспорту МК \cite{}. На контакты питания VDDA и VDD3P3 необходимо подать напряжение в диапазоне 2.3-3.6 В, GND подключить к общему проводу и для 
фильтрации на каждый вход добавим конденсатор GRM21BR71C105K на 1 мкФ.

Также, рекомендуется соединить контакты CAP1 и СAP2 согласно паспорту МК \cite{}. Это делается для 
стабилизации работы МК. Схема соединения представлена на рисунке \ref{fig::CAP1CAP2}

\img[htb]{fig::CAP1CAP2}{CAP1CAP2.png}{Подключение контактов CAP1 и CAP2}{0.4}
