\section {Выбор вентиляторов для охлаждения}

\textbf{Расчет необходимого потока воздуха для охлаждения серверного шкафа}

Для определения мощности вентиляторов и расчета потока воздуха, необходимого для охлаждения серверного шкафа, необходимо учесть следующие параметры:

\begin{itemize}
    \item Размеры серверного шкафа: 1.3 м (высота) $\times$ 1 м (ширина) $\times$ 0.8 м (глубина).
    \item Объем серверного шкафа: 
    \[
    V = 1.3 \times 1 \times 0.8 = 1.04 \, \text{м}^3
    \]
    \item Тепловыделение оборудования серверного шкафа такого типа составляет: 1300-1500 Вт.
    \item Допустимое изменение температуры: 5°C.
\end{itemize}

\textbf{Расчет необходимого потока воздуха}

Необходимый поток воздуха можно рассчитать по следующей формуле:

\[
\text{Q [}frac{\text{м}^3}{\text{ч}}\text{]} = \frac{P_max \times 2.982}{\Delta T}
\]

где:
\begin{itemize}
    \item Вт — тепловыделение оборудования в ваттах.
    \item $\Delta T$ — допустимое изменение температуры в градусах Цельсия.
    \item $2.982$ — коэффициент, полученный опытным путем.
\end{itemize}

Подставим известные значения:

\[
\text{Q} = \frac{1500 \times 2.982}{5} \approx 894.6 \, \frac{\text{м}^3}{\text{ч}}
\]

Таким образом, для эффективного охлаждения серверного шкафа потребуется вентилятор или несколько вентиляторов с суммарной производительностью около 894.6 м³/мин.

\textbf{Выбор вентиляторов}

Для охлаждения возьмем 4 вентилятора, необходимая мощность которых рассчитаем следующим образом:

\[
 n = \frac{894.6}{4} \approx 223.65
\]

Таким образом, потребуется 4 вентилятора > 223.65 м³/ч для достижения необходимого потока воздуха (с небольшим запасом).

\textbf{Выбор вентиляторов}

Выберем вентилятор SUNON EEC0381B1-А99, размеры которого показаны на рисунке \ref{fig::Fan}.

\img[ht]{fig::Fan}{Fan.png}{Размеры вентилятора SUNON EEC0381B1-А99}{0.8}

Основные характеристики которого представлены в таблице~\ref{tab::FanCharacteristics}.

\begin{table}[h!]
	\centering
	\caption{Основные характеристики вентилятора EEC0381B1-А99}
	\begin{tabular}{| m{7cm} | m{7cm} |}
		\hline
		\textbf{Характеристика} & \textbf{Значение} \\ \hline
			Модель устройства & EEC0381B1-А99 \\ \hline
			Размеры устройства & 120 мм × 120 мм × 38 мм \\ \hline
			Напряжение питания & 6~13.8В DC \\ \hline
			Ток потребления & 800мА \\ \hline
			Скорость вращения & 3100 об/мин (RPM) \\ \hline
			Воздушный поток & 234.4 м³/ч \\ \hline
			Уровень шума & 48 dBA \\ \hline
			Рабочая температура & -10°C до +70°C \\ \hline
	\end{tabular}
	\label{tab::FanCharacteristics}
\end{table}

Как видно из таблицы, мощность воздушного потока составляет 234.4~м³/ч, что удовлетворяет основному требованию. Управление осуществляется через транзистор . 

Для подключения и управления вентилятора к микроконтроллеру, используем N-канальный MOSFET транзистор.

Для ограничения тока через затвор MOSFET и защиты выходного пина микроконтроллера ESP32 был выбран резистор номиналом 220~Ом, а также диод для защиты от обратного тока. Такая схема позволит эффективно управлять скоростью вентилятора с использованием ШИМ сигнала.

Выберем транзистор IRLZ44N. Его основные параметры представлены в таблице \ref{tab::MosfetCharacteristics}

\begin{table}[h!]
	\centering
	\caption{Характеристики транзистора IRLZ44N}
	\begin{tabular}{| m{7cm} | m{7cm} |}	
		\hline
			$I_{D}$ &  47A (при 25°C)\\ \hline
			$V_{DS}$ & 55В \\ \hline
			$R_{DS}$ & 22 мОм (при $V_{GS}$ = 10В) \\ \hline
			$V_{GS}$(th) & 1.0В - 2.0В (пороговое напряжение затвора) \\ \hline
	\end{tabular}
	\label{tab::MosfetCharacteristics}
\end{table}

Также, для полного подключения необходимы:

\begin{enumerate}

	\item Резистор для затвора с номиналом в 220~Ом --- необходим для ограничения тока и защиты микроконтроллера;
	
	\item Диод для защиты от обратного тока 1N5819 (Шоттки диод, выдерживающий ток до 1А и имеющий низкое падение напряжения) --- необходим для защиты от скачков напряжения, возникающих при выключении вентилятора.

\end{enumerate}
