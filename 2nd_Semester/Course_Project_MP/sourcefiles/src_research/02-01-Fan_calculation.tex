\chapter {Выбор вентиляторов для охлаждения}

\section{Расчет необходимого потока воздуха для охлаждения серверного шкафа}

Для определения мощности вентиляторов и расчета потока воздуха, необходимого для охлаждения серверного шкафа, необходимо учесть следующие параметры:

\begin{itemize}
    \item Размеры серверного шкафа: 1.3 м (высота) $\times$ 1 м (ширина) $\times$ 0.8 м (глубина).
    \item Объем серверного шкафа: 
    \[
    V = 1.3 \times 1 \times 0.8 = 1.04 \, \text{м}^3
    \]
    \item Тепловыделение оборудования серверного шкафа такого типа составляет: 1300-1500 Вт.
    \item Допустимое изменение температуры: 5°C.
\end{itemize}

\textbf{Расчет необходимого потока воздуха}

Необходимый поток воздуха можно рассчитать по следующей формуле:

\[
\text{Q [}frac{\text{м}^3}{\text{ч}}\text{]} = \frac{P_max \times 2.982}{\Delta T}
\]

где:
\begin{itemize}
    \item Вт — тепловыделение оборудования в ваттах.
    \item $\Delta T$ — допустимое изменение температуры в градусах Цельсия.
    \item $2.982$ — коэффициент, полученный опытным путем.
\end{itemize}

Подставим известные значения:

\[
\text{Q} = \frac{1500 \times 2.982}{5} \approx 894.6 \, \frac{\text{м}^3}{\text{ч}}
\]

Таким образом, для эффективного охлаждения серверного шкафа потребуется вентилятор или несколько вентиляторов с суммарной производительностью около 894.6 м³/мин.

\textbf{Выбор вентиляторов}

Для охлаждения возьмем 4 вентилятора, необходимая мощность которых рассчитаем следующим образом:

\[
 n = \frac{894.6}{4} \approx 223.65
\]

Таким образом, потребуется 4 вентилятора > 223.65 м³/ч для достижения необходимого потока воздуха (с небольшим запасом).

Вентилятор SUNON EEC0381B1-G99 обеспечивает 234.4 м³/ч.

