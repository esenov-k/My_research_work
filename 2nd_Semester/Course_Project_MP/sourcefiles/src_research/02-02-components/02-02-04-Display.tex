\section {Выбор дисплея}

Для отображения информации о температуре, а также о влажности и давлении, выберем дисплей, на который будет выводится вся информация. 

Чтобы вся информация была отражена в полном объеме, выберем OLED экран с разрешением 128 на 128 пикселей от компании WaveShare. Параметры дисплея приведены в таблице \ref{tab::specifications}

\begin{table}[h!]
	\centering
	\caption{Технические характеристики OLED дисплея}
	\begin{tabular}{| m{7cm} | m{7cm} |}
		\hline
		Рабочее напряжение & 3.3B-5B \\ \hline
		Тип дисплея & OLED \\ \hline
		Разрешение & 128 × 128 пикселя \\ \hline
		Размер пикселя & 0.185 × 0.185 мм \\ \hline
		Интерфейс & 4-wire SPI / I2C \\ \hline
		Размер дисплея & 26.86 × 26.86 мм \\ \hline
	\end{tabular}
	
	\label{tab::specifications}
\end{table}

Подключение к микроконтроллеру осуществляется при помощи семипинового коннектора, который показан на рисунке \ref{fig::OLED_Pin}. Значение пинов показано в таблице \ref{tab::pinout}

\img[ht]{fig::OLED_Pin}{OLED_Pin.png}{Распиновка дисплея}{0.8}

\begin{table}[htb]
	\centering
	\caption{Назначение выводов OLED дисплея}
	\begin{tabular}{| m{2cm} | m{10cm} |}
		\hline
		\textbf{Вывод} & \textbf{Назначение} \\ \hline
		VCC & Питание (вход 3.3V / 5V) \\ \hline
		GND & Земля \\ \hline
		DIN & Вход данных \\ \hline
		CLK & Вход тактового сигнала \\ \hline
		CS & Выбор микросхемы, активный низкий уровень \\ \hline
		DC & Выбор данных/команд (высокий уровень для данных, низкий для команд) \\ \hline
		RST & Сброс, активный низкий уровень \\ \hline
		\end{tabular}
	\label{tab::pinout}
\end{table}

Как видно из рисунка \ref{fig::OLED_Pin}, для использования протокола I2C, необходимо соединить коннекторы BS1 и BS2 на 1, а пины CLK и DIN использовать как SCL и SDA соответственно. Таким образом, для подключения экрана к микроконтроллеру, необходимо 4 провода: VCC, GND, CLK (SCL), DIN (SDA).

Таким образом, получаем экран для вывода всей необходимой информации для контроля температуры, влажности и, при желании, давления, на протоколе I2C. Так как в случае использования такого протокола необходимы подтягивающие резисторы, выберем их номинал исходя из общепринятых вариантов --- 10 кОм.
