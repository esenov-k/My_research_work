\section {Выбор потенциометра}

Для настройки точного значения нижнего порога температуры, будем использовать потенциометр на 10кОм.

Выберем потенциометр фирмы BOURNS, модель PTV09 \cite{datasheet::Potentiometr}. Он обладает высокой износоустойчивостью в 10000 циклов, углеродным резистивным элементом, а также довольной широкой рабочей температурой от -10°C до +50°C. Основные характеристики потенциометра приведены в таблице \ref{tab::PotentiometerCharact}.

\begin{table}[h!]
	\centering
	\caption{Характеристики потенциометра}
	\begin{tabular}{| m{6cm} | m{8cm} |}
		\hline
		Стандартный диапазон сопротивлений & 1 кОм до 1 МОм \\ \hline
		Остаточное сопротивление & 500 Ом или ±1\% макс. \\ \hline
		Рабочая температура & -10°C до +50°C \\ \hline
		Мощность & 0.05 Вт\\ \hline
		Максимальное рабочее напряжение & 20 В DC, 50 В AC (линейный и аудио) \\ \hline
		Шум при скольжении & 100 мВ макс. \\ \hline
		Механический угол & 280° ±10° \\ \hline
		Ресурс вращения & 10,000 циклов \\ \hline
	\end{tabular}
	\label{tab::PotentiometerCharact}
\end{table}