\section {Выбор микроконтроллера}

Выбор МК будет осуществлен на основе следующих критериев:

\begin{itemize}

	\item Популярность МК --- популярные МК являются хорошо изученными, имеют большую аудиторию и, чаще всего, имеют хорошую документацию;
	
	\item Достаточная мощность для работы всех компонентов;
	
	\item Достаточное количество пинов для подключения дополнительных устройств для модернизации платы;

	\item Небольшая цена и доступность.
	
\end{itemize}

На основе выставленных критериев был выбран МК ESP32-D0WD-V3, показанный на рисунке \ref{fig::ESP32}. Использование этого МК для управления кулерами является перспективным решением, так как этот МК обладает высокой производительностью, поддержкой Wi-Fi и Bluetooth, а также множеством входов/выходов для подключения периферийных устройств. ESP32 позволяет реализовать гибкую систему мониторинга и управления температурой в серверных системах, обеспечивая эффективное охлаждение и снижение энергозатрат.

\img[htb]{fig::ESP32}{ESP32.png}{Внешний вид МК ESP32-D0WD-V3}{0.5}

\begin{table}[h!]
	\centering
	\caption{Характеристики датчика BME280}
	\begin{tabular}{| m{9cm} | m{6cm} |}
		\hline
		Параметр & Значение \\ \hline
		Разрядность & 32 бит \\ \hline
		ROM & 448 кб \\ \hline
		SRAM & 520 кб \\ \hline
		Энергопотребление в спящем режиме & < 1 мА\\ \hline
		Напряжение питания & 2.3 - 3.6 В \\ \hline
	\end{tabular}
	\label{tab::TempSensorCharacteristics}
\end{table}