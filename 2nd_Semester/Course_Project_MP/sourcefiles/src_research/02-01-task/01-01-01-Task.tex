Адекватный воздушный поток жизненно важен для обеспечения эксплуатационной надежности критически важных 
элементов внутри шкафа. Охлаждение и воздушный поток должны быть достаточно хорошими, чтобы предотвратить 
накопление тепла (горячие точки) и гарантировать, что все компоненты в серверном шкафу должным образом охлаждаются независимо от того, где они размещены (внизу, в середине или вверху стойки).

Рассмотрим несколько типов охлаждения:

\begin{itemize}

	\item Охлаждение естественной конвекцией --- форма охлаждения, основанная на обмене тепла с окружающей средой. Если воздух вокруг серверной стойки холоднее внутренней температуры, тепло внутри шкафа будет естественным образом излучаться через боковые стороны и двери, и внутренняя температура соответственно снизится. Данный тип охлаждения слабо эффективен.
	
	\item Принудительное конвекционное охлаждение --- вентиляторы устанавливаются внутри серверной стойки, что усиливает воздушный поток и уменьшает барьер термического сопротивления между шкафом и окружающей средой, что в целом помогает устранять горячие точки и понижать общую температуру
	серверного шкафа.
	
	\item Активное охлаждение --- подразделяется на два подтипа:
	
	\begin{itemize}
		
		\item Активное воздушное охлаждение --- вместо охлаждения всей комнаты встраивается кондиционер в серверный шкаф. Эта конфигурация образует замкнутую систему, которые более эффективны, так как они ориентированы на охлаждение одного шкафа.
		
		\item Активное жидкостное охлаждение --- в системах этого типа используется специальная жидкость для охлаждения воздуха внутри шкафа, горячего/холодного коридора или всего объекта обработки данных.
	
	\end{itemize}

\end{itemize}

В рамках курсового проекта рассмотрим серверный шкаф со следующими размерами: 1.3 м (высота) $\times$ 1 м (ширина) $\times$ 0.8 м (глубина). Примерный вид серверного шкафа показан на рисунке \ref{fig::ServerBox}.

\img[htb]{fig::ServerBox}{ServerBox.png}{Вид напольного серверного шкафа серии ШТК-М}{0.4}

Таким образом, в рамках курсового проекта необходимо:

\begin{enumerate}

	\item Рассчитать параметры для компонентов охлаждения серверного шкафа;
	
	\item Сформировать требования к МК и выбрать его;
	
	\item Выбрать датчик температуры и давления;
	
	\item Разработать конструкцию печатной платы и создать для нее следующую конструкторскую 
	документацию:
	
	\begin{itemize}
	
		\item Спецификацию; 
		
		\item Сборочный чертеж;
		
		\item Структурная схема;
		
		\item Перечень элементов;
		
		\item Чертеж печатной платы;
		
		\item Электрическая принципиальная схема.
	
	\end{itemize}
	
\end{enumerate}