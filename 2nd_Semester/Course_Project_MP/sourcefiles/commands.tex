% В данном файле предлагается располагать пользовательские команды

% Удобная команда для вставки картинок
\newcommand{\img}[5][htb]{
  	\begin{figure}[#1]
    		\centering
    		\includegraphics[width = #5\linewidth]{./img/#3}
   	 	\caption{#4}
	 	\label{#2}
  	\end{figure}
}

% Включение сквозной нумерации рисунков
\counterwithout{figure}{chapter}

% Изменение команды из шаблона bmstu для корректной вставки номера страницы
% списка литературы
\renewcommand{\makebibliography}
{	
\printbibliography[heading=bibintoc,title={СПИСОК ИСПОЛЬЗОВАННЫХ ИСТОЧНИКОВ}]
}

% Команда для добавления сокращения
\newcommand{\abbr}[2]{
	\DeclareAcronym{#1}{
    	short={#1},
    	long={--- #2},
    }
}

% Команда для вставки списка сокращений. По умолчанию выводится сортированный
% список сокращений. Если необходимо вывести список в порядке указания (в файле
% abbreviations.tex), то необходимо указать параметр sort=false.
\newcommand{\printabbr}{
	\printacronyms[display=all,sort=true,name={ОБОЗНАЧЕНИЯ И СОКРАЩЕНИЯ},
	preamble={В	настоящей расчетно-пояснительной записке применяют следующие 
	сокращения и обозначения.}] 
	\addcontentsline{toc}{chapter}{ОБОЗНАЧЕНИЯ И СОКРАЩЕНИЯ}
}


% Вставка титульника для курсового проекта. Пример вставки:
%\makecourseprojecttitle{Специальное машиностроение}                             % Факультет
%                {Робототехнические системы и мехатроника}                       % Название кафедры
%                {Разработка робота-пылесоса}                                    % Тема КП
%                {М.А. Козлов/СМ7-23М}                                           % Группа студента
%                {В.В. Зеленцов}                                                 % Научный руководитель
%                {}                                                              % Консультант, если есть

\newcommand{\makecourseprojecttitle}[6]
{
	\documentmeta{РПЗ к КР}{#4}{}{#3}

	\begin{titlepage}
		\centering

		\titlepageheader{#1}{#2}
		\vspace{15.8mm}

		\titlepagenotetitle{К КУРСОВОМУ ПРОЕКТУ}{#3}
		\vfill

		\titlepageauthors{#4}{Руководитель курсового проекта}{#5}{#6}{}
		\vspace{14mm}

		\textit{{\the\year} г.}
	\end{titlepage}

	\setcounter{page}{2}
}

% Вставка титульника для реферата. Пример использования:
%\makeabstracttitle{Социальные и гуманитарные науки}                             % Факультет
%                {Философия}                                                     % Название кафедры
%                {Методология технических наук}                                  % Тема реферата
%		 {М.А. Козлов/СМ7-23М}                                           % Фамилия и группа студента
%                {В.А. Иноземцев}   	                                         % Преподаватель

\newcommand{\makeabstracttitle}[5]
{
	\documentmeta{Реферат}{#4}{}{#3}

	\begin{titlepage}
		\centering

		\titlepageheader{#1}{#2}
		\vspace{15.8mm}

		\titlepageabstracttitle{#3}
		\vfill

		\titlepageauthors{#4}{Преподаватель}{#5}{}{}
		\vspace{14mm}

		\textit{{\the\year} г.}
	\end{titlepage}

	\setcounter{page}{2}
}

% Служебная команда вставки основной части титульника
\newcommand{\titlepageabstracttitle}[1]
{
	{
		\LARGE \bfseries
		РЕФЕРАТ \\
	}
	\vspace{5mm}
	{
		\Large \itshape
		\vspace{3mm}
		НА ТЕМУ: \\
	}
	{
		\Large \itshape
		\vspace{5mm}
		<<#1>> \\
	}
}
