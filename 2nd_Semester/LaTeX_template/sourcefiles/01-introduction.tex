\chapter*{ВВЕДЕНИЕ}
\addcontentsline{toc}{chapter}{ВВЕДЕНИЕ}

На основании плана ВКРМ, который был составлен в предыдущей НИР, в этом семестре 
работа будет посвящена разработке системы детекции нескольких объектов в 
реальном времени. Такая система является одним из основных инструментов, применяемых
в мультиобъектном тренинге. Данная технология позволяет определять какие объекты
окружают автономное транспортное средство во время его движения и, в зависимости от
этого объекта, система подбирает алгоритм действий в различных ситуациях. 

Основным требованием к системе реального времени является обработка не менее 25 кадров 
в секунду (fps).

Таким образом, можно сформулировать цель и задачи на научно-исследовательскую работу. 

Целью данной научно-исследовательской работы является разработка системы детекции нескольких объектов в реальном времени для автономного транспорта..

Задачи, решаемые в рамках научно-исследовательской следующие:

\begin{itemize}

	\item Выбрать нейросеть для детекции нескольких объектов;

	\item Выбрать набор данных, который хорошо подойдет для обучения в сфере автономного вождения;
	
 	\item Обучить выбранную модель нейросети на данных с выбранного набора данных;
	
	\item Проанализировать полученный результат, сделать выводы.
	
\end{itemize}

