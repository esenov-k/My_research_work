\chapter*{ЗАКЛЮЧЕНИЕ}
\addcontentsline{toc}{chapter}{ЗАКЛЮЧЕНИЕ}

По результатам обучения модели можно сделать следующие выводы:

\begin{enumerate}

	\item mAP = 0.832 – обучение модели прошло хорошо, средняя точность довольно высока;
	\item FPS = 29-52, что удовлетворяет требованию;
	\item Ошибки и ложные срабатывания возникают при:
	
	\begin{enumerate}
		\item Высокой скорости транспортных средств;
		\item Высокая загруженность сцены;
		\item Недостаточная освещенность.
	\end{enumerate}
	
\end{enumerate}

Для устранения ошибок и ложных срабатываний можно провести дополнительное обучение с 
увеличением вариативности данных для повышения точности детекции и уменьшения количества 
FP результатов, внедрить механизмы аугментации данных для обучения, чтобы повысить
устойчивость модели к изменяющимся условиям окружающей среды и изменить параметры обучения.

В целом, модель показывает обнадеживающие результаты и предоставляет сильную базу для дальнейшей оптимизации системы детекции для автономного транспорта.

Таким образом, в ходе НИР были решены все поставленные задачи:

\begin{itemize}

	\item Была выбрана нейросеть YOLOv8 в качестве системы детекции объектов в реальном
	времени для автономного транспорта;
	\item Выбран набор данных KITTI, который удовлетворяет всем предъявленным критериям;
	\item Проведено обучение модели для дальнейшего применения в сфере автономного 
	транспорта;
	\item По полученным результатам обучения был проведен анализ и сделаны выводы, а также
	были представлены возможные способы улучшения показателей. 
	
\end{itemize}